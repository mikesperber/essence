\documentclass{article}

\usepackage{hyperlatex}
\usepackage{palatino}
\renewcommand{\ttdefault}{cmtt}
\usepackage{xspace}
\usepackage{alltt}
\W\newenvironment{alltt}{\begin{rawhtml}<p><pre>\end{rawhtml}}{\begin{rawhtml}</pre></p>\end{rawhtml}}
\usepackage{url}
\W\newcommand{\url}[1]{\xlink{\texttt{#1}}{#1}}

\newcommand{\essenceversion}{1.2}

\W\newcommand{\langle}{<}
\W\newcommand{\rangle}{>}

\W\newcommand{\bigskip}{\begin{rawhtml}<p>\end{rawhtml}}

\newcommand{\meta}[1]{{\noindent\mbox{\textrm{$\langle$#1$\rangle$}}}}
\newcommand{\hyper}[1]{\meta{#1}}
\newcommand{\hyperi}[1]{\hyper{#1$_1$}}
\newcommand{\hyperii}[1]{\hyper{#1$_2$}}
\newcommand{\hyperj}[1]{\hyper{#1$_i$}}
\newcommand{\hypern}[1]{\hyper{#1$_n$}}
\renewcommand{\var}[1]{\noindent\mbox{\textit{#1}}}
\newcommand{\vari}[1]{\var{#1$_1$}}
\newcommand{\varii}[1]{\var{#1$_2$}}
\newcommand{\variii}[1]{\var{#1$_3$}}
\newcommand{\variv}[1]{\var{#1$_4$}}
\newcommand{\varj}[1]{\var{#1$_j$}}
\newcommand{\varn}[1]{\var{#1$_n$}}

\W\newcommand{\frenchspacing}{}
\W\newcommand{\textnormal}[1]{#1}

\newcommand{\defining}[1]{{\emph{#1}}}
\newcommand{\ide}[1]{{\frenchspacing\textnormal{\texttt{#1}}}}

\newcommand{\exprtype}{syntax}

\W\newcommand{\unskip}{}
\W\newcommand{\textrm}[1]{#1}

\W\newenvironment{flushleft}{}{}
\W\newcommand{\hfill}{}
\W\newcommand{\fill}{}
\W\newcommand{\linebreak}{\\}
\W\newcommand{\hspace}[1]{ }
\W\newcommand{\hspace*}[1]{ }

% Primitive prototype
\newcommand{\pproto}[2]{\unskip%
\mbox{\texonly{\spaceskip=0.5em}#1}%
\mbox{ }\texonly{\nobreak}\htmlonly{ }\textrm{#2}}

% Parenthesized prototype
\newcommand{\proto}[3]{\par\bigskip\begin{flushleft}\pproto{(\texttt{#1}}{\textit{#2})}\hspace*{\fill}{#3}\end{flushleft}}

% Variable prototype
\newcommand{\vproto}[2]{\pproto{#1}{#2}}

% Random prototype
\newcommand{\rproto}[3]{\proto{#1}{#2}{#3}}

\newcommand{\constant}[1]{\texttt{#1}}

\newcommand{\dotsfoo}{\ldots\texonly{\thinspace}}

\newcommand{\codefont}[1]{\texttt{#1}}

\T\newcommand{\filename}[1]{\url{#1}}
\W\newcommand{\filename}[1]{\texttt{#1}}

\htmltitle{Essence---An LR Parser Generator for Scheme}
\htmldirectory{html}
\htmladdress{\begin{rawhtml}<a href="http:/www.deinprogramm.de/sperber/">Mike
  Sperber</a>, <a
  href="http://www.informatik.uni-freiburg.de/~thiemann/">Peter Thiemann</a>\end{rawhtml}}

\title{Essence---An LR Parser Generator for Scheme\\
  \normalsize{Version \essenceversion}}
\author{\xlink{Mike Sperber}{http://www.deinprogramm.de/sperber/}\\\xlink{\texttt{sperber@deinprogramm.de}}{mailto:sperber@deinprogramm.de}\\
  \xlink{Peter
    Thiemann}{http://www.informatik.uni-freiburg.de/~thiemann/}\\\xlink{\texttt{thiemann@informatik.uni-freiburg.de}}{mailto:thiemann@informatik.uni-freiburg.de}
  }

\begin{document}

\maketitle

\begin{abstract}
  \noindent
  Essence is a generator for LR(k) and SLR(k) parsers in Scheme.  The
  generated parsers perform error recovery, and are highly efficient.
  Testing and debugging a parser does not require an
  edit---generate---compile---test cycle.  Rather, the parser
  generator results from a general parser which takes the input
  grammar as a parameter; no generation and recompilation is necessary
  to try out changes to a grammar.  The generated parsers result from
  the general parser by an automatic program transformation called
  \emph{partial
    evaluation}~\cite{SperberThiemann1995-pepm,SperberThiemann2000}.
  This guarantees consistency and ensures correctness.  However, no
  specific knowledge of partial evaluation is required to use Essence.
\end{abstract}
%
This document assumes elementary knowledge about S-attributed
grammars and LR parsing, available in almost any compiler
construction
textbook~\cite{Chapman1987,SippuSoisalon-Soininen1990,AhoSethiUllman1986,WilhelmMaurer1995}.
It may also be helpful to study the documentation of more traditional
parser generation packages such as Yacc~\cite{Johnson1975} or
Bison~\cite{DonnellyStallman1995}.

\section{Introduction}
\label{sec:introduction}

The pragmatics of using Essence are slightly different from that of
using other parser or parser generation packages.

At the heart of parsing is, as usual, a context-free grammar.  Essence
provides a new syntactic form
\link{\codefont{define-grammar}}{form:define-grammar} which embeds a
language for attributed context-free grammars into Scheme.  This is
different from other parser generation packages which either represent 
a context-free grammar as an S-expression object or in some special
syntax in a special file.

Given a grammar, parsing can proceed in one of two modes:
%
\begin{itemize}
\item A general \codefont{parse} procedure will accept a grammar, a
  parsing method (SLR or LR), a lookahead size, and an input, and
  produce the result of parsing and attribute evaluation.  This mode
  of operation allows instant turnaround, but also parses very
  slowly.  It is good for incremental development, but impractical for
  production parsers.
\item A parser generator (automatically generated from the
  implementation of \codefont{parse}) produces a specialized parser
  from a grammar, a parsing method, and a lookahead.  The specialized
  parser only accepts an input as an argument, but is otherwise
  identical in operation to the general parser.  It is highly
  efficient, and therefore good for production use.
\end{itemize}

\section{Prerequisites}
\label{sec:preliminaries}

Currently, Essence itself only runs under
Scheme~48~\cite{KelseyRees1995}.  However, the generated parsers run
under any R$^5$RS~\cite{KelseyClingerRees1998} Scheme.  Moreover, the
dependencies on non-standard features of Scheme48 which Essence uses
have been carefully factored out with the help of the Scheme~48 module
system; making it work in any given Scheme implementation is not hard.

The Essence distribution contains four files to be loaded into
Scheme~48's configuration package: \filename{interfaces.scm},
\filename{packages.scm}, and \filename{generator-packages.scm}.  To
make Essence available within Scheme~48, it is always necessary to
load \filename{interfaces.scm} (via \codefont{,config ,load
  interfaces.scm}).  To run the generic parsers, it is necessary to
also load \filename{packages.scm}, to use the parser generator,
\filename{src/cps-lr-genext.config.scm} and
\filename{generator-packages.scm}.

\section{Grammars}
\label{sec:grammars}

Context-free grammars are at the heart of parser generation.  Essence
allows specifying so-called \emph{S-attributed grammars} with
evaluation rules for synthesized attributes.  The assumption is that
each node in the parse tree carries an instance of exactly one
synthesized attribute, and an Essence grammar provides an expression describing 
how to compute the attribute along with each production.

The structure \codefont{essence-grammars} provides a defining form for
grammars.  (It also provides numerous accessors and algorithms over
grammars.  However, these are not relevant for using Essence.)

An Essence grammar consists of two data objects: a representation of
the grammar itself, and an enumeration (see \filename{doc/big-scheme.txt}
in the Scheme~48 distribution for a description of the enumeration
facility in Scheme~48) which is needed to symbolically encode the
input to the parser.  The macro \codefont{define-grammar} from the
\codefont{essence-grammars} structure form defines both:

\label{form:define-grammar}
\proto{define-grammar}{ \meta{variable$_1$} \meta{variable$_2$}\hfill\linebreak\texonly{\hspace*{1em}}
    \meta{terminals}
  \meta{start-symbol} \meta{rules}}{syntax}

\paragraph{Syntax:} \meta{Terminals} has the form
%
\begin{alltt}
  (\meta{terminal} \dotsfoo)
\end{alltt}
%
where each \meta{terminal} is an \meta{identifier}.  
\meta{Start-symbol} must
be a an identifier.  \meta{Rules} has the form:
%
\begin{alltt}
  (\meta{rule} \dotsfoo)
\end{alltt}
%
where each indivial \meta{rule} has the form
%
\begin{alltt}
  (\meta{nonterminal} ((\meta{grammar-symbol} \dotsfoo) \meta{attribution}) \dotsfoo)
\end{alltt}
%
where each \meta{grammar-symbol} is either a \meta{nonterminal}, a
\meta{terminal}, or \codefont{\$error}.  (The latter is for directing
\link{error recovery}[ (see
Sec.~\ref{sec:error-recovery})]{sec:error-recovery} \meta{Attribution}
is a Scheme expression.

The set of nonterminals is defined by the \meta{nonterminals}s of the
\meta{rule}s.  The nonterminals must be disjoint from the terminals.
Moreover, lThe start symbol must be a nonterminal.

\paragraph{Semantics:} \codefont{Define-grammar} defines a
context-free grammar along with an enumeration type for its symbols.
\codefont{Define-grammar} binds a data object representing the grammar 
to its first argument, and an enumeration type for its symbols to its
second argument.

The third argument to \codefont{define-grammar} is a list of
nonterminals, the fourth a list of terminals.  The fifth argument is
the start symbol (which must be one of the nonterminals), then comes a
list of the grammar rules.

A grammar rule specifies a list of productions for the specified
nonterminal.  Each subform \texttt{((\meta{grammar-symbol} \dotsfoo)
  \meta{attribution})}
specifies a right-hand side and an attribution.

The attribution is a Scheme expression, which may have free variables
\codefont{\$}$i$, where $i$ ranges from $1$ to the number of symbols
on the right-hand side of the production.  During parsing, the Essence
parser binds \codefont{\$}$i$ to the attribute instance of the $i$th
symbol on the right-hand-side when evaluating the attribution.

Here is a simple example grammar for arithmetic expressions:
%
\begin{verbatim}
(define-grammar g10 g10-symbol
  (+ - * / l r n)
  E
  ((E ((T) $1)
      ((T + E) (+ $1 $3))
      ((T - E) (- $1 $3)))
   (T ((P) $1)
      ((P * T) (* $1 $3))
      ((P / T) (/ $1 $3)))
   (P ((n) $1)
      ((l E r) $2))))
\end{verbatim}
%
This definition establishes an enumeration type \codefont{g10-symbol}
with components (in that order):
%
\label{list:grammar-enum}
%
\begin{itemize}
\item \codefont{\$Start} (for the fresh start symbol generated by
  \codefont{define-grammar}),
\item the nonterminals in the same order as in the
  \codefont{define-grammar} form,
\item \codefont{\$Error}, and
\item the terminals in the same order as the \codefont{define-grammar}
  form.
\end{itemize}

\section{Running a Parser}
\label{sec:running-a-parser}

Parsing with respect to a grammar does not require generating a
specialized parser along with the associated overhead of compiling and
loading.  Essence provides general parsers which accept a grammar as
input and parse ``right away.''  This allows incremental debugging and 
development of attributed grammars to be used with Essence.

Essence actually comes with a number of different implementations of
LR parsing.  The \filename{packages.scm} configuration file contains
definitions for a range of structures all with the interface
\codefont{essence-parser-interface}.  The one intended for production use is in the
\codefont{essence-cps-lr} structure whose implementation resides in
\filename{src/cps-lr.scm}.  \codefont{Essence-parser-interface} describes only 
one binding called \codefont{parse}:

\label{proc:parse}
\proto{parse}{grammar lookahead method trace-level input}{procedure}
%
\begin{itemize}
\item \var{Grammar} is a grammar defined by \codefont{define-grammar}.
\item \var{Lookahead} is a non-negative integer denoting the
  lookahead the parser uses.
\item \var{Method} is a symbol, either \codefont{lr} or
  \codefont{slr}, specifying the parsing method used---either full LR
  parsing or SLR parsing.
\item \var{Trace-level} is 0, 1, 2, 3 depending on the amount of
  tracing desired.  0 means no tracing, 1 means that the closure of
  the state where an error is encountered is passed to
  \ide{parse-error}.  2 means that the states encountered during
  parsing are printed, 3 means that the complete closures are printed.
\item \var{Input} is a list of pairs; each pair consists of an enumerand of the
  terminals of \var{grammar} and the corresponding attribute value.  It is the
  input to the parser. 
  
  The enumerands are the \emph{ordinal numbers} of the enumerated
  values.  These are easiest obtained via the \codefont{enumerated} structure
  of Scheme~48.  In the absence of an implementation of \codefont{grammar} and
  \codefont{enumerated}, the \link{list of components}[ in the previous
  section]{list:grammar-enum} describes the mapping between grammar symbols and
  enumerands; the enumerands are 0-based.
\end{itemize}
%
\ide{Parse} returns the result of attribute evaluation on the parse
tree induced by \var{input}.  This is ultimately the result of the
attribution associated with the start production.

If the grammar contains productions containing \codefont{\$error}
symbols, the parser will attempt \link{error recovery}[ (see
Sec.~\ref{sec:error-recovery})]{sec:error-recovery} when possible.

\section{Generating a Specialized Parser}
\label{sec:parser-generation}

In addition to simply calling \codefont{parse}, Essence also allows
the generation of highly efficient specialized parsers with respect to
a grammar, lookahead, and parsing method.  Essence offers a (Unix)
batch version of the parser generator, as well as a Scheme~48 package
which allows access from within a REPL.

In order to run, specialized parsers require an \codefont{parse-error}
procedure.  The specialized parser will call \codefont{parse-error} when an
unrecoverable error occurs.  It has the following signature:

\proto{parse-error}{message maybe-closure symbol input}{procedure}
%
Here, \var{maybe-closure} is either \verb|#f| or the LR closure in
which the error occurred, depending on the tracing level.
\var{Symbol} is the symbol on which the parser tried to shift, and
\var{input} is the remaining input.

\subsubsection*{Batch operation}
Installation of Essence creates a binary called \filename{essence}.
When called with a \codefont{--help} or \codefont{-h} argument, it
prints a synopsis of its syntax:
%
\begin{verbatim}
essence ( -g goal-proc | --goal-proc=goal-proc | --goal-procedure=goal-proc )
        ( -m method | --method=method )
        ( -l lookahead | -lookahead=lookahead )
        input-file grammar-name output-file
\end{verbatim}
%
\begin{itemize}
\item \texttt{Input-file} is a Scheme source file which Essence will
  load into a package with the standard R$^5$RS Scheme bindings as
  well as \link{\codefont{define-grammar}}{form:define-grammar}.  The
  file must contain the definition for at least one grammar.
\item \texttt{Grammar-name} is the name of the grammar defined in
  \texttt{input-file} for which Essence is to generate a specialized
  parser.
\item \texttt{Output-file} is the name of the file into which Essence
  writes the specialized parser.
\item \texttt{Goal-proc} is the name of entry procedure into the
  specialized parser.  It will accept just one argument, an input
  list.  The default is \texttt{parse}.
\item \texttt{Method} is the parsing method, either \texttt{slr} or
  \texttt{lr}.  The default is \texttt{slr}.
\item \texttt{Lookahead} is the lookahead size, a non-negative
  number.  The default is \texttt{1}.
\end{itemize}
%
In addition to the \texttt{goal-proc} procedure, the output file also
contains a \texttt{define-\linebreak[0]enumeration} form that defines the mapping
between grammar symbols and enumeration values.  The form has the
following syntax:

\proto{define-enumeration}{ \meta{identifier} (\meta{symbol} \ldots)}

The identifier is the name of the enumeration type of the grammar, and
the \meta{symbol}s are the names of all terminals and nonterminals of
the grammar (including \texttt{\$error}).  The first \meta{symbol} is
mapped to enumeration value 0, the second to 1, and so forth.  The
\texttt{define-enumeration} form is suitable for use with Scheme~48's
\texttt{enumerated} package.

\subsubsection*{REPL operation}
The \filename{generator-packages.scm} configuration file defines a
structure \codefont{essence-cps-lr-generate} which offers a procedure that
generates specialized parsers:

\proto{generate-parser}{grammar lookahead method goal-name}{procedure}
%
\var{Grammar}, \var{lookahead}, and \var{method} are as with
\link{\codefont{parse}}{proc:parse}.

The \var{goal-name} argument to \ide{generate-parser} is a symbol
which names the entry procedure into the parser.
\ide{Generate-parser} generates a list of S-expressions which, when
written out sequentially, represent the code of the specialized
parser.

\section{Error Recovery}
\label{sec:error-recovery}

Essence parsers can perform recovery from parsing errors in the manner
of Yacc~\cite{Johnson1975} and Bison~\cite{DonnellyStallman1995}.  The
basic idea is that the author of a grammar can specify special
error productions at critical places in a grammar designed to ``catch''
parsing errors.  This allows printing specially tailored error
messages as well as some control over attribute evaluation in such a
case.

Error productions contain a special grammar symbol \codefont{\$error}
on the right-hand side.  (\codefont{\$Error} must not be explicitly
declared as a terminal or nonterminal in the \codefont{define-grammar}
form.)

When an error occurs during parsing, an Essence parser pretends that
it has just seen \codefont{\$error} in the input.  It will go back to
the last LR state capable of accepting \codefont{\$error} as the next
symbol in the input.  Moreover, it discards terminals from the input
until the next input terminal is acceptable as the next input symbol
\emph{after} it has consumed \codefont{\$error}.  Subsequently, the
parser resumes work as usual.

To prevent excessive avalanching of error messages, the parser makes
sure that it consumes at least three input terminals after an
attempted error recovery until it attempts recovery again.

Here is an example for the constant arithmetic expressions grammar
guaranteed to catch all errors:
%
\begin{verbatim}
(define-grammar g10-error g10-error-symbol
  (+ - * / l r n)
  E
  ((E ((T) $1)
      (($error) 0)
      ((T + E) (+ $1 $3))
      ((T - E) (- $1 $3)))
   (T ((P) $1)
      ((P * T) (* $1 $3))
      ((P / T) (/ $1 $3)))
   (P ((n) $1)
      ((l E r) $2)
      ((l $error r) 0))))
\end{verbatim}
%
Apart from the first catch-all rule containing \codefont{\$error}, the
parser will also, when encountering an error inside a parenthesized
expression, skip until the next closing parenthesis to resume parsing.


\section{Example Session}
\label{sec:example-session}

To see all this in action, we work through a little example involving one of
the provided example grammars. First, to start the system, type
\begin{verbatim}
% scheme48 -i essence.image -h 8000000
\end{verbatim}
at the shell prompt. The mechanism to define a grammar are available from
structure \codefont{essence-grammars} (see~\ref{sec:grammars}). To open the structure
type
\begin{verbatim}
> ,open essence-grammars
\end{verbatim}
to the Scheme48 system, followed by
\begin{verbatim}
> ,load examples/toy-grammars.scm
\end{verbatim}
to load the definitions for some simple grammars.

Loading the corresponding inputs requires enumerated values, hence
\begin{verbatim}
> ,open enumerated
> ,load examples/toy-inputs.scm
\end{verbatim}
defines the example inputs.

Open the parser module by typing
\begin{verbatim}
> ,open cps-lr
\end{verbatim}
to gain access to the \codefont{parse}
function (see Sec.~\ref{sec:running-a-parser}). 

As a sample run, consider the grammar \codefont{g10} which specifies arithmetic 
expressions. The terminals \codefont{l} and \codefont{r} stand for opening
and closing brackets, whereas \codefont{n} stands for a number.
\begin{verbatim}
> (parse g10 1 'lr 0 i10-1)
147
\end{verbatim}

To specialize a parser requires to open the structure
\codefont{essence-cps-lr-generate}:
\begin{verbatim}
> ,open essence-cps-lr-generate
\end{verbatim}
The \codefont{generate-parser} function from this structure (see
Sec.~\ref{sec:parser-generation}) performs the specialization:
\begin{verbatim}
> (generate-parser g10 1 'lr 'expr-parser)
\end{verbatim}

To perform the same specialization task via the command line interface type
\begin{verbatim}
% ./essence -g expr-parser -m lr -l 1 \
            examples/toy-grammars.scm g10 /tmp/expr-parser.scm
\end{verbatim}

\bibliographystyle{plain} 
\bibliography{abbrevs,papers,books,theses,misc,collections}
\end{document}
